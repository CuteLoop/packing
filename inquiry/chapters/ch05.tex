\chapter{Optimizing the Square Size: Feasibility and Joint Optimization}

Up to this point, simulated annealing has been used to optimize circle positions
inside a \emph{fixed} square of side length $L$.
We now address a more fundamental geometric question:

\begin{quote}
What is the smallest square side length $L^\star$ for which $N$ identical circles
of radius $r$ can be packed without overlap?
\end{quote}

This chapter develops two complementary solver strategies:

\begin{enumerate}
\item \textbf{Feasibility search}: treat $L$ as a parameter, test feasibility via SA,
      and search for the minimal feasible $L$.
\item \textbf{Joint optimization}: include $L$ directly in the optimization state
      and minimize a soft energy that trades off feasibility and size.
\end{enumerate}

Both approaches are built using the same inquiry-driven workflow:
\[
\text{Geometry} \;\longrightarrow\; \text{Invariant} \;\longrightarrow\;
\text{Test} \;\longrightarrow\; \text{Code}.
\]

Throughout, we distinguish between:
\begin{itemize}
\item \emph{guaranteed bounds} derived from geometry, and
\item \emph{heuristic behavior} introduced by stochastic optimization.
\end{itemize}

%--------------------------------------------------------------------
\section{Why Optimizing $L$ Is Hard}

Packing problems are inherently nonconvex.
Even for fixed $L$, feasibility is not a simple constraint satisfaction problem;
it must be discovered algorithmically.

Optimizing $L$ adds a second layer of difficulty:
\begin{itemize}
\item reducing $L$ increases boundary pressure,
\item smaller $L$ amplifies overlap penalties,
\item feasibility becomes a probabilistic outcome of the inner solver.
\end{itemize}

We therefore separate the problem into two solver paradigms.

%--------------------------------------------------------------------
\section{Approach I: Feasibility Search}

\subsection{Problem Formulation}

For a fixed $L$, define \emph{feasibility} as:
\[
E_{\text{pair}}(X) = 0 \quad\text{and}\quad E_{\text{wall}}(X,L) = 0.
\]

Numerically, we test:
\[
E_{\text{pair}}(X) \le \varepsilon, \qquad
E_{\text{wall}}(X,L) \le \varepsilon,
\]
for a small tolerance $\varepsilon$.

The feasibility solver asks:
\begin{quote}
For which values of $L$ does simulated annealing find a feasible configuration?
\end{quote}

This motivates an outer search in $L$.

%--------------------------------------------------------------------
\subsection{Guaranteed Lower Bounds}

Before running any optimization, geometry alone already constrains $L$.

\subsubsection{Diameter Bound}

At minimum, a single circle must fit:
\[
L \ge 2r.
\]
This is a trivial but unavoidable constraint.

\subsubsection{Area Bound}

The total area of all circles is:
\[
A_{\text{circles}} = N\pi r^2.
\]
Since the square has area $L^2$, any feasible packing must satisfy:
\[
L^2 \ge N\pi r^2
\quad\Rightarrow\quad
L \ge r\sqrt{N\pi}.
\]

This bound is \emph{necessary} and therefore guaranteed.

\subsubsection{Density-Aware Bound (Optional)}

The maximal packing density of equal circles in the plane is
\[
\delta_{\max} = \frac{\pi}{2\sqrt{3}} \approx 0.9069.
\]
Even ignoring boundary effects,
\[
L^2 \ge \frac{N\pi r^2}{\delta_{\max}}
\quad\Rightarrow\quad
L \ge r\sqrt{\frac{N\pi}{\delta_{\max}}}.
\]

Boundary effects only reduce achievable density, so this remains a valid lower bound.

\paragraph{Combined lower bound.}
In practice, we use:
\[
L_{\text{lo}} = \max\!\left(2r,\; r\sqrt{N\pi}\right),
\]
optionally replacing the area bound with the density-aware bound.

%--------------------------------------------------------------------
\subsection{Guaranteed Upper Bounds}

Lower bounds alone are insufficient; feasibility search also requires
a value of $L$ that is \emph{known} to be feasible.

\subsubsection{Grid Construction}

A simple constructive packing places circles on a square grid
with spacing $2r$:
\[
x_{ij} = \bigl(r + 2ri,\; r + 2rj\bigr).
\]

Let
\[
k = \lceil\sqrt{N}\rceil.
\]
Then a $k\times k$ grid fits inside a square of side length:
\[
L_{\text{grid}} = 2rk.
\]

This configuration has:
\[
E_{\text{pair}} = 0, \qquad E_{\text{wall}} = 0,
\]
and therefore provides a \emph{constructive feasibility witness}.

\paragraph{Upper bound.}
We set:
\[
L_{\text{hi}} = 2r\lceil\sqrt{N}\rceil.
\]

This bound is crude but guaranteed and deterministic.

%--------------------------------------------------------------------
\subsection{Outer Bisection on $L$}

With a bracket $[L_{\text{lo}}, L_{\text{hi}}]$ in hand, we perform
an outer bisection search.

\begin{enumerate}
\item Set $L_{\text{mid}} = \tfrac12(L_{\text{lo}} + L_{\text{hi}})$.
\item Run simulated annealing at fixed $L_{\text{mid}}$.
\item If a feasible configuration is found:
      \begin{itemize}
      \item record the witness,
      \item set $L_{\text{hi}} \leftarrow L_{\text{mid}}$.
      \end{itemize}
\item Otherwise, set $L_{\text{lo}} \leftarrow L_{\text{mid}}$.
\end{enumerate}

Because simulated annealing is stochastic, feasibility is a
\emph{probabilistic predicate}. To reduce false negatives,
the solver may retry each $L_{\text{mid}}$ with multiple deterministic seeds.

The algorithm terminates when:
\[
L_{\text{hi}} - L_{\text{lo}} \le \text{tolerance}.
\]

%--------------------------------------------------------------------
\section{Approach II: Joint Optimization of $(X,L)$}

An alternative is to optimize positions and box size simultaneously.

\subsection{Energy Model}

We define a soft objective:
\[
E(X,L) = E_{\text{pair}}(X) + E_{\text{wall}}(X,L) + \alpha L,
\]
with $\alpha > 0$ controlling the pressure to shrink the square.

Key properties:
\begin{itemize}
\item $E(X,L) \ge 0$ always,
\item feasibility corresponds to $E_{\text{pair}} = E_{\text{wall}} = 0$,
\item reducing $L$ is rewarded only when geometry allows it.
\end{itemize}

\subsection{Joint Proposal Kernel}

The state space is now $(X,L)$.
At each SA step:
\begin{itemize}
\item with probability $p$, propose a move of a single circle position,
\item with probability $1-p$, propose a move in $L$:
      \[
      L' = \max(2r,\; L + \delta_L), \qquad \delta_L \sim \mathcal{N}(0,\sigma_L^2).
      \]
\end{itemize}

Acceptance is handled by the same Metropolis rule using the correct $\Delta E$.

\subsection{Interpretation}

Joint optimization replaces hard feasibility with a soft competition:
\begin{itemize}
\item early at high temperature, $L$ can shrink aggressively,
\item overlaps and wall violations are temporarily tolerated,
\item at low temperature, only feasible shrinkage survives.
\end{itemize}

Unlike feasibility search, this approach produces a \emph{single trajectory}
rather than an outer–inner loop, but its outcome depends sensitively on $\alpha$
and the proposal scales.

%--------------------------------------------------------------------
\section{Comparison of the Two Approaches}

\begin{center}
\begin{tabular}{lcc}
\toprule
 & Feasibility Search & Joint Optimization \\
\midrule
Guarantees & Explicit feasibility & Soft, parameter-dependent \\
Bounds & Required & Not required \\
Complexity & Outer $\times$ inner & Single SA run \\
Interpretability & Clear witnesses & Continuous trade-off \\
Sensitivity & Low (with retries) & High ($\alpha$, $\sigma_L$) \\
\bottomrule
\end{tabular}
\end{center}

In practice:
\begin{itemize}
\item feasibility search is preferable when correctness and reproducibility matter;
\item joint optimization is useful for exploratory or heuristic searches.
\end{itemize}

%--------------------------------------------------------------------
\section{Checkpoint}

At the end of this chapter, you should be able to:

\begin{itemize}
\item derive guaranteed lower and upper bounds for $L^\star$;
\item construct feasibility witnesses without optimization;
\item implement an outer bisection loop driven by simulated annealing;
\item understand the trade-offs of joint $(X,L)$ optimization;
\item distinguish mathematical guarantees from heuristic behavior.
\end{itemize}

In the next chapter, we revisit these solvers from a gradient-based perspective
and contrast stochastic annealing with deterministic optimization methods.
