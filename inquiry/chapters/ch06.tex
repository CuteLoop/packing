

\chapter{Discrete Formulations of Packing via Independent Sets}
\label{chap:discrete-packing-mis}

\section{From Continuous Packing to Discrete Candidates}

We begin with a classical continuous packing problem: given a compact container
$\Omega \subset \mathbb{R}^2$ (e.g.\ a square or disk) and identical circles of radius $r$,
place as many circles as possible inside $\Omega$ without overlap.

In the continuous setting, the configuration space is infinite-dimensional,
and the problem is highly nonconvex. To make algorithmic progress, we introduce
a \emph{discretization of configuration space}.

\subsection{Candidate placements}

Let $\mathcal{P} = \{p_1, \dots, p_M\} \subset \Omega$ be a finite set of candidate
circle centers. These may arise from:
\begin{itemize}
\item a uniform Cartesian grid,
\item a union of rotated grids to reduce anisotropy,
\item or a local patch extracted from a continuous solver.
\end{itemize}

Each candidate $p_i$ represents the placement of a circle centered at $p_i$.

A candidate is \emph{feasible} if the entire disk lies inside the container:
\[
\operatorname{dist}(p_i, \partial \Omega) \ge r.
\]

Only feasible candidates are retained.

\subsection{Conflict detection}

Two candidates $p_i$ and $p_j$ are said to be \emph{in conflict} if placing circles
at both locations would cause overlap:
\[
\|p_i - p_j\| < 2r.
\]

This pairwise condition captures all geometric constraints of the packing problem
once the candidate set has been fixed.

\section{Graph Construction}

The discrete packing problem can now be represented as a graph.

\begin{definition}[Conflict Graph]
Let $G = (V,E)$ be a graph where:
\begin{itemize}
\item Each vertex $i \in V$ corresponds to a candidate placement $p_i \in \mathcal{P}$.
\item An edge $(i,j) \in E$ exists if and only if $p_i$ and $p_j$ are in conflict.
\end{itemize}
\end{definition}

The graph $G$ encodes all pairwise incompatibilities between candidate placements.
Importantly, this graph is:
\begin{itemize}
\item geometric,
\item sparse (conflicts are local),
\item independent of the optimization method used later.
\end{itemize}

\section{Packing as an Independent Set Problem}

A valid packing corresponds to a selection of candidates such that no two
selected placements conflict.

\begin{definition}[Independent Set]
A subset $S \subset V$ is an \emph{independent set} if no two vertices in $S$
share an edge.
\end{definition}

\subsection{Maximum vs.\ maximal independent sets}

Two notions are relevant:
\begin{itemize}
\item A \emph{maximal} independent set cannot be extended by adding another vertex.
\item A \emph{maximum} independent set has the largest possible cardinality.
\end{itemize}

In packing problems, we are interested in the \emph{maximum independent set} (MIS),
since its cardinality corresponds to the maximum number of circles that can be placed
using the candidate set.

\begin{problem}[Discrete Packing via MIS]
Given a conflict graph $G = (V,E)$, find
\[
\max_{S \subset V} |S| \quad ext{such that } S ext{ is an independent set.}
\]
\end{problem}

This formulation is exact for the discretized problem and separates geometry
(from graph construction) from combinatorial optimization.

\section{Integer Linear Programming Formulation}

The maximum independent set problem admits a standard integer programming formulation.

\subsection{Binary decision variables}

Introduce variables
\[
x_i \in \{0,1\}, \quad i \in V,
\]
where $x_i = 1$ indicates that candidate $p_i$ is selected.

\subsection{MILP formulation}

The packing problem becomes:
\begin{align}
\max \quad & \sum_{i \in V} x_i \\
ext{subject to} \quad
& x_i + x_j \le 1, \quad \forall (i,j) \in E, \\
& x_i \in \{0,1\}, \quad \forall i \in V.
\end{align}

Each constraint $x_i + x_j \le 1$ enforces non-overlap for a conflicting pair.
This formulation is exact and captures all geometric constraints implicitly
through the graph.

\section{LP Relaxation and Its Interpretation}

Solving the MILP directly is computationally expensive.
A standard relaxation replaces integrality by bounds:
\[
x_i \in [0,1].
\]

\subsection{LP relaxation}

The relaxed problem is:
\begin{align}
\max \quad & \sum_{i \in V} x_i \\
ext{subject to} \quad
& x_i + x_j \le 1, \quad \forall (i,j) \in E, \\
& 0 \le x_i \le 1.
\end{align}

This linear program provides:
\begin{itemize}
\item an upper bound on the true packing number,
\item fractional solutions that encode local packing density.
\end{itemize}

In geometric settings, fractional values often highlight regions of high
structural order even before integrality is enforced.

\section{Algorithmic Solution via Constraint Generation}

The full conflict graph may contain a large number of edges.
However, most constraints are never active in optimal solutions.

\subsection{Lazy constraint generation}

An efficient strategy is to generate constraints iteratively:

\begin{enumerate}
\item Start with variables $x_i \in [0,1]$ and a minimal constraint set.
\item Solve the LP.
\item Detect violated conflict constraints:
\[
x_i + x_j > 1 \quad ext{for some conflicting pair } (i,j).
\]
\item Add the violated constraints.
\item Repeat until no violations remain.
\end{enumerate}

Because conflicts are local, the number of active constraints remains manageable.

\subsection{Transition to integer solutions}

Once the LP relaxation stabilizes, integrality constraints are reinstated:
\[
x_i \in \{0,1\}.
\]

The resulting problem is solved using a \emph{branch-and-cut} strategy:
\begin{itemize}
\item LP relaxation provides bounds,
\item branching enforces integrality,
\item constraint generation continues as needed.
\end{itemize}

This approach yields exact solutions for local patch problems and provides
certificates of optimality.

\section{Interpretation and Scope}

This discrete formulation has several important properties:
\begin{itemize}
\item It cleanly separates geometry from optimization.
\item It applies equally to circles, polygons, and more general shapes.
\item It is well suited for \emph{local patch analysis}, where structure
such as hexagonal order can emerge without being prescribed.
\end{itemize}

However, the method does not scale to large global packings and is best
used as a:
\begin{itemize}
\item verification tool,
\item local structure discovery method,
\item or subproblem within a hybrid continuous--discrete pipeline.
\end{itemize}

In later chapters, we will combine this discrete machinery with continuous
relaxation and periodic boundary conditions to study bulk packing structure.
