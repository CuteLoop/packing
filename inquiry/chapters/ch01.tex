\chapter{Simulated Annealing for Circle Packing in a Square}

\section*{Guiding Question}
How can a stochastic, temperature--driven algorithm reliably solve deterministic, highly non--convex geometric optimization problems where gradient--based methods fail?

This chapter establishes the \emph{theoretical and conceptual foundation} for simulated annealing (SA) through an inquiry--based lens. We use the geometric problem of packing $N$ identical circles into a square of minimal side length as a concrete running example. The goal is not to implement SA yet, but to understand \emph{why it works}, \emph{what assumptions it relies on}, and \emph{which design choices matter} before turning to test--driven C implementations in later chapters.

\section{The Optimization Problem}

We consider the problem of packing $N$ circles of radius $r$ into a square of side length $L$, minimizing $L$ subject to non--overlap and boundary constraints.

\subsection{Configuration Space}
A configuration is described by
\[
x = (x_1,y_1,\dots,x_N,y_N) \in \mathbb{R}^{2N},
\]
with constraints
\[
\lvert(x_i,y_i)-(x_j,y_j)\rvert \ge 2r, \quad i\neq j, \qquad r \le x_i,y_i \le L-r.
\]

The feasible set is a subset of $\mathbb{R}^{2N}$ defined by pairwise distance constraints and box constraints.

\section{Why Classical Optimization Fails}

\subsection{Inquiry}
\begin{itemize}
\item Is the feasible set convex?
\item Is the objective function differentiable everywhere?
\item How does the number of local minima scale with $N$?
\end{itemize}

Each question reveals a structural obstruction:
\begin{itemize}
\item The feasible set is \emph{highly non--convex}.
\item Contact events between circles introduce \emph{nonsmoothness}.
\item The number of metastable configurations grows combinatorially with $N$.
\end{itemize}

These are not numerical pathologies but geometric ones. Gradient--based methods fail \emph{by design}, not merely by poor tuning.

\section{From Optimization to Sampling}

The key conceptual shift of simulated annealing is to replace deterministic descent with \emph{probabilistic exploration}.

\subsection{Energy Formulation}
We introduce a scalar energy
\[
E(x,L) = E_{\text{overlap}}(x) + \alpha L,
\]
where $E_{\text{overlap}}$ penalizes violations of geometric constraints and $\alpha>0$ balances feasibility against compactness.

\subsection{Key Question}
Why should we ever accept a move that increases $E$?

The answer is central: rejecting all uphill moves traps the algorithm in local minima. Allowing energy--increasing moves enables escape from metastable states.

\section{Statistical Mechanics Perspective}

\chapter{Simulated Annealing for Circle Packing in a Square}

\section*{Guiding Question}
How can a stochastic, temperature--driven algorithm reliably solve deterministic, highly non--convex geometric optimization problems where gradient--based methods fail?

This chapter establishes the \emph{theoretical and conceptual foundation} for simulated annealing (SA) through an inquiry--based lens. We use the geometric problem of packing $N$ identical circles into a square of minimal side length as a concrete running example. The goal is not to implement SA yet, but to understand \emph{why it works}, \emph{what assumptions it relies on}, and \emph{which design choices matter} before turning to test--driven C implementations in later chapters.

\section{The Optimization Problem}

We consider the problem of packing $N$ circles of radius $r$ into a square of side length $L$, minimizing $L$ subject to non--overlap and boundary constraints.

\subsection{Configuration Space}
A configuration is described by
\[
x = (x_1,y_1,\dots,x_N,y_N) \in \mathbb{R}^{2N},
\]
with constraints
\[
\lvert(x_i,y_i)-(x_j,y_j)\rvert \ge 2r, \quad i\neq j, \qquad r \le x_i,y_i \le L-r.
\]

The feasible set is a subset of $\mathbb{R}^{2N}$ defined by pairwise distance constraints and box constraints.

\section{Why Classical Optimization Fails}

\subsection{Inquiry}
\begin{itemize}
\item Is the feasible set convex?
\item Is the objective function differentiable everywhere?
\item How does the number of local minima scale with $N$?
\end{itemize}

Each question reveals a structural obstruction:
\begin{itemize}
\item The feasible set is \emph{highly non--convex}.
\item Contact events between circles introduce \emph{nonsmoothness}.
\item The number of metastable configurations grows combinatorially with $N$.
\end{itemize}

These are not numerical pathologies but geometric ones. Gradient--based methods fail \emph{by design}, not merely by poor tuning.

\section{From Optimization to Sampling}

The key conceptual shift of simulated annealing is to replace deterministic descent with \emph{probabilistic exploration}.

\subsection{Energy Formulation}
We introduce a scalar energy
\[
E(x,L) = E_{\text{overlap}}(x) + \alpha L,
\]
where $E_{\text{overlap}}$ penalizes violations of geometric constraints and $\alpha>0$ balances feasibility against compactness.

\subsection{Key Question}
Why should we ever accept a move that increases $E$?

The answer is central: rejecting all uphill moves traps the algorithm in local minima. Allowing energy--increasing moves enables escape from metastable states.

\section{Statistical Mechanics Perspective}

Simulated annealing is grounded in equilibrium statistical mechanics.

\subsection{Gibbs Measure}
For temperature $T>0$, define
\[
\pi_T(x) = Z_T^{-1} e^{-E(x)/T},
\]
where $Z_T$ is the normalizing constant.

\begin{itemize}
\item High $T$: broad exploration of configuration space.
\item Low $T$: concentration near global minima.
\end{itemize}

Optimization emerges as the zero--temperature limit of sampling.

\subsection{Metropolis Acceptance Rule}
Given a current state $x$ and a proposal $x'$, accept $x'$ with probability
\[
p = \min\bigl(1, e^{-(E(x')-E(x))/T}\bigr).
\]

\subsection{Inquiry}
\begin{itemize}
\item Why does this rule preserve $\pi_T$ as an invariant distribution?
\item What role does detailed balance play?
\end{itemize}

These questions will later justify the correctness of the algorithm independent of implementation.

\section{Simulated Annealing as a Limit Process}

Simulated annealing proceeds by lowering the temperature:
\[
T_0 > T_1 > \cdots > T_k \to 0.
\]

\subsection{Theoretical Guarantee}
With logarithmic cooling, $T_k \sim c/\log k$, simulated annealing converges almost surely to a global minimizer. Although impractical, this result explains \emph{why} the method can work at all.

\section{Algorithmic Ingredients}

Every simulated annealing algorithm consists of:
\begin{enumerate}
\item \textbf{State space}: configurations $x$ (and possibly $L$).
\item \textbf{Energy}: objective plus penalties.
\item \textbf{Proposal kernel}: how candidate states are generated.
\item \textbf{Acceptance rule}: Metropolis criterion.
\item \textbf{Cooling schedule}: temperature decay.
\end{enumerate}

Failure of any component compromises the method.

\section{Energy Design for Circle Packing}

Hard constraints destroy ergodicity. Instead we use soft penalties.

\subsection{Pairwise Overlap Penalty}
For distance $d$ between circle centers,
\[
\phi(d) =
\begin{cases}
(2r-d)^2, & d < 2r,\\
0, & d \ge 2r.
\end{cases}
\]

Total overlap energy:
\[
E_{\text{overlap}}(x) = \sum_{i<j} \phi(|x_i-x_j|) + \sum_i \phi_{\text{wall}}(x_i).
\]

\subsection{Inquiry}
\begin{itemize}
\item Why quadratic penalties instead of infinite barriers?
\item How does penalty scaling interact with temperature?
\end{itemize}

\section{Proposal Mechanisms}

Common proposal strategies include:
\begin{itemize}
\item local Gaussian perturbations of a single circle,
\item occasional global reshuffling,
\item temperature--dependent step sizes.
\end{itemize}

A key heuristic is
\[
	\textbf{proposal scale} \propto \sqrt{T}.
\]

\section{Optimizing the Square Size}

Two approaches are common:

\subsection{Feasibility Search}
Fix $L$, test feasibility via SA, and perform an outer binary search on $L$.

\subsection{Joint Optimization}
Optimize $(x,L)$ jointly using
\[
E(x,L) = E_{\text{overlap}}(x) + \alpha L.
\]

\subsection{Inquiry}
\begin{itemize}
\item Why must $\alpha$ be temperature--aware?
\item What failure modes arise if $L$ shrinks too early?
\end{itemize}

\section{Common Failure Modes}

\begin{center}
\begin{tabular}{ll}
	\textbf{Symptom} & \textbf{Cause} \\
\hline
Freezing & Cooling too fast \\
Jittering & Proposal scale too large \\
Poor minima & Bad initialization \\
Slow convergence & Poor energy scaling \\
\end{tabular}
\end{center}

\section{Exercises}

\begin{enumerate}
\item Prove detailed balance for the Metropolis acceptance rule.
\item Show that hard constraints break ergodicity.
\item Compare simulated annealing and basin hopping on small $N$.
\item Design a temperature--dependent penalty function.
\item Explain why SA only superficially resembles stochastic gradient descent.
\end{enumerate}

\section{Conceptual Summary}

Simulated annealing replaces deterministic descent with controlled stochastic exploration. Its validity comes from statistical mechanics; its effectiveness comes from careful algorithmic design. In later chapters, this conceptual foundation will be translated into \emph{test--driven C implementations}, where correctness precedes performance.
