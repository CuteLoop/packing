\documentclass[11pt]{article}

\usepackage{amsmath, amssymb}
\usepackage{graphicx}
\usepackage{booktabs}
\usepackage{hyperref}
\usepackage{geometry}
\usepackage{caption}
\usepackage{subcaption}

\geometry{margin=1in}

\title{A Diagnostic Smoke Test for Simulated Annealing\\
       in 2D Circle Packing}
\author{ }
\date{\today}

\begin{document}
\maketitle

\section{Purpose and Scope}

This report documents a \emph{diagnostic smoke run} of a correctness-first
simulated annealing (SA) implementation for packing identical circles
in a square container.
The goal of the run is not optimality or performance, but
\textbf{observability}:
to verify that the algorithm behaves consistently with theoretical
expectations and to understand how temperature, proposals, and penalties
interact over time.

All quantities shown here are produced automatically by the codebase,
including per-step logging and post-processing plots.

\section{Problem Setup}

We consider packing $N$ identical circles of radius $r$ in a square
container of side length $L$.
A configuration is given by
\[
X = \{x_i \in \mathbb{R}^2 : i=1,\dots,N\}.
\]

The total energy is
\[
E(X,L) = E_{\text{pair}}(X) + E_{\text{wall}}(X,L) + \alpha L,
\]
where $E_{\text{pair}}$ penalizes overlaps and $E_{\text{wall}}$
penalizes boundary violations.
In this baseline run, $L$ is fixed and $\alpha=0$.

\subsection{Run Parameters}

The smoke run reported here uses:
\begin{center}
\begin{tabular}{ll}
\toprule
Number of circles & $N = 20$ \\
Circle radius & $r = 1.0$ \\
Box side length & $L = 20.0$ \\
Initial temperature & $T_0 = 2.0$ \\
Cooling factor & $\gamma = 0.99995$ \\
Number of steps & $n_{\text{steps}} = 8\times 10^5$ \\
Proposal scale & $\sigma = 0.30$ \\
Random seed & $123$ \\
\bottomrule
\end{tabular}
\end{center}

The run is designed to complete in under five minutes on typical hardware.

\section{Initial Conditions}

\subsection{Initial Position}

The initial configuration $X_0$ is generated randomly using a seeded
pseudo-random number generator.

If $L - 2r > 0$, each coordinate is sampled independently from the
interior window:
\[
x_i \sim \mathrm{Uniform}(r, L-r), \qquad
y_i \sim \mathrm{Uniform}(r, L-r).
\]

Otherwise (degenerate case), sampling falls back to
\[
x_i, y_i \sim \mathrm{Uniform}(0, L).
\]

Thus, the initialization is \textbf{uniform, interior, and deterministic}
given the seed.

\subsection{Initial Box Length}

The box side length is set by the input parameter $L$ and remains
fixed throughout the run.
For this experiment,
\[
L = 20.0.
\]

No dynamic resizing of the container is performed in this baseline.

\section{Annealing Schedule and Runtime}

\subsection{Algorithmic Duration}

The algorithm executes exactly $n_{\text{steps}}$ Metropolis proposals.
In this run,
\[
n_{\text{steps}} = 8\times 10^5.
\]

\subsection{Computational Cost}

Each proposal updates exactly one circle.
The incremental energy computation scales as
\[
\mathcal{O}(N),
\]
so the total computational work scales approximately as
\[
\mathcal{O}(n_{\text{steps}} \cdot N).
\]

\subsection{Cooling Schedule}

A geometric cooling schedule is used:
\[
T_{t+1} = \gamma T_t,
\qquad
T_t = T_0 \gamma^t.
\]

For this run,
\[
T_0 = 2.0, \qquad \gamma = 0.99995.
\]

The temperature half-life is
\[
t_{1/2} = \frac{\ln(1/2)}{\ln(\gamma)} \approx 13{,}860 \text{ steps},
\]
which yields a slow, steady cooling over the duration of the run.

\section{Logged Quantities and Diagnostics}

At every Metropolis step, the following quantities are logged:

\begin{itemize}
\item step index,
\item total energy $E$,
\item temperature $T$,
\item pair overlap energy $E_{\text{pair}}$,
\item wall penalty energy $E_{\text{wall}}$,
\item acceptance indicator (0/1),
\item index of the proposed circle.
\end{itemize}

This full-resolution logging enables detailed post-hoc analysis.

\section{Results and Plots}

\subsection{Energy, Temperature, Acceptance, and Penalties}

Figure~\ref{fig:main-panels} shows four diagnostic panels side by side:

\begin{enumerate}
\item Total energy and best-so-far energy versus step.
\item Temperature versus step (log scale).
\item Rolling acceptance rate.
\item Pair and wall energy components versus step.
\end{enumerate}

\begin{figure}[h]
\centering
\includegraphics[width=\textwidth]{../../sa_circlepack/out/sa_better_smoke/sa_better_smoke_plots.png}
\caption{Main diagnostic panels for the simulated annealing run.}
\label{fig:main-panels}
\end{figure}

As expected:
\begin{itemize}
\item $E_{\text{pair}}$ drops rapidly early, indicating resolution of overlaps.
\item $E_{\text{wall}}$ reflects boundary interactions and stabilizes later.
\item The total energy decreases monotonically in its best-so-far envelope.
\item The acceptance rate decreases as temperature cools.
\end{itemize}

\subsection{Energy Change Distribution}

Figure~\ref{fig:de-hist} shows the distribution of
\[
\Delta E_t = E(t) - E(t-1).
\]

Early in the run, the distribution is broad, while late in the run it
concentrates near zero, reflecting freezing of the dynamics.

\begin{figure}[h]
\centering
\includegraphics[width=0.7\textwidth]{../../sa_circlepack/out/sa_better_smoke/sa_better_smoke_dE.png}
\caption{Distribution of energy changes $\Delta E$.}
\label{fig:de-hist}
\end{figure}

\subsection{Moved Index Histogram}

Figure~\ref{fig:moved-hist} shows the histogram of proposed circle indices.
The distribution is approximately uniform, confirming unbiased selection.

\begin{figure}[h]
\centering
\includegraphics[width=0.7\textwidth]{../../sa_circlepack/out/sa_better_smoke/sa_better_smoke_moved.png}
\caption{Histogram of moved circle indices.}
\label{fig:moved-hist}
\end{figure}

\section{Role of Temperature}

Temperature affects \emph{only} the acceptance of uphill moves.
For $\Delta E > 0$,
\[
\mathbb{P}(\text{accept}) = \exp(-\Delta E / T).
\]

Thus:
\begin{itemize}
\item High $T$: frequent uphill acceptance $\Rightarrow$ exploration.
\item Low $T$: rare uphill acceptance $\Rightarrow$ exploitation and freezing.
\end{itemize}

Importantly, temperature does \emph{not} modify the proposal distribution.
Proposals are always drawn from
\[
\delta \sim \mathcal{N}(0, \sigma^2 I),
\]
with fixed $\sigma$.
Temperature acts as a probabilistic filter on these proposals.

\section{Discussion and Recommendations}

With aggressive cooling (e.g.\ $\gamma=0.999$), temperature halves quickly,
and the algorithm becomes greedy early.
The slower cooling used here provides better separation between exploration
and exploitation phases and produces more interpretable diagnostics.

Possible next steps include:
\begin{itemize}
\item adaptive proposal scaling,
\item dynamic logging stride,
\item joint optimization of $(X,L)$,
\item incremental logging of $\Delta E$.
\end{itemize}

\section{Conclusion}

This smoke run confirms that the simulated annealing implementation
behaves consistently with theoretical expectations.
The instrumentation allows direct interpretation of geometric and
stochastic effects, providing a solid foundation for further optimization
and scaling.

\end{document}
